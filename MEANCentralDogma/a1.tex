\input{tex/preamble}
\graphicspath{ {./figures/} }


\title{%
    \begin{center} \includegraphics[width=150pt]{dna} \end{center}
    A1: \textit{The MEAN Central Dogma}}
\author{Julian Mazzitelli, \textit{iGEM UofT} 
}
\date{May 14, 2015 \\ \vspace{5pt}
{\small Due: May 20, 2015 or May 22, 2015, or \textit{anytime thereafter} }}

\hypersetup{
    colorlinks=true,
    urlcolor=TealBlue
}

% hrefs
\def\AngularJS{\href{https://angularjs.org/}{AngularJS}}
\def\RESTful{\href{http://en.wikipedia.org/wiki/Representational_state_transfer}{RESTful}}
\def\NodeJS{\href{https://nodejs.org/}{Node.js}}
\def\ExpressJS{\href{http://expressjs.com/}{Express}}
\def\API{\href{http://en.wikipedia.org/wiki/Application_programming_interface}{API}}
\def\MongoDB{\href{https://www.mongodb.org/}{MongoDB}}
\def\SPA{\href{http://en.wikipedia.org/wiki/Single-page_application}{SPA}}
\def\centraldogma{\href{http://en.wikipedia.org/wiki/Central_dogma_of_molecular_biology}{central dogma}}
\def\pythondna{\href{https://github.com/igemuoftATG/tutorials/tree/master/Python/\%5B03\%5DDNA!}{Python DNA!}}
\def\javascripting{\href{https://github.com/sethvincent/javascripting}{javascripting}}
\def\learnyounode{\href{https://github.com/workshopper/learnyounode}{learnyounode}}
\def\expressworks{\href{https://github.com/azat-co/expressworks}{expressworks}}

\begin{document}
\reversemarginpar

\maketitle

\section{Introduction}

This assignment will be an introduction to the modern full-stack web application
while also promoting an improved understanding of the transfer of information
within biological systems. You will build a single page application (\SPA) with
\AngularJS\ which communicates with a \RESTful\ \NodeJS\ (\ExpressJS)
\textit{application programming interface} (\API) and stores data on a \MongoDB\
database. This is the MEAN stack:

\begin{center}
    MongoDB $\leftrightarrow$ Node.js (Express) $\leftrightarrow$ AngularJS
\end{center} 

Moreover, you will write code in more than one language, interact with a number
of frameworks/APIs, make a new best friend (documentation!), and consider the
viability of your chosen data structures and algorithms for a given problem. 

Your API will be capable of manipulating data forwards and backwards between
each stage of biological information flow.  You will not know what format the
incoming data is, and will have to decide which conversion to use.  That is,
given DNA, RNA, or a polypeptide sequence, you must convert to each of the other
two.  Ultimately, you will describe the \centraldogma\ of molecular biology:

\begin{center}
\begin{tikzpicture}[scale=0.3]
\tikzstyle{every node}+=[inner sep=0pt]
\draw [black] (19.9,-32.7) circle (3);
\draw (19.9,-32.7) node {$DNA$};
\draw [black] (39.3,-32.7) circle (3);
\draw (39.3,-32.7) node {$RNA$};
\draw [black] (59.7,-32.4) circle (3);
\draw (59.7,-32.4) node {$protein$};
\draw [black] (57.86,-34.767) arc (-41.41295:-137.72331:24.22);
\fill [black] (21.78,-35.04) -- (21.94,-35.97) -- (22.68,-35.29);
\draw [black] (22.9,-32.7) -- (36.3,-32.7);
\fill [black] (36.3,-32.7) -- (35.5,-32.2) -- (35.5,-33.2);
\draw [black] (42.3,-32.66) -- (56.7,-32.44);
\fill [black] (56.7,-32.44) -- (55.89,-31.96) -- (55.91,-32.96);
\end{tikzpicture}
\end{center}


\mpnote

\noindent We will not consider DNA/DNA, RNA/RNA, or \textbf{protein/DNA}
interactions in this assignment. Though you should realize the \textbf{crucial} 
implications of positive/negative regulation of DNA expression through
protein/DNA interactions.\\

\mymarginpar{\textit{Prerequisites}}

\noindent You \textbf{must} complete our \pythondna\ assignment before starting
this. You will call your python script from the back-end. You should also
have completed \javascripting,  \learnyounode, and \expressworks. If you have
trouble completing the DNA! assignment, consider trying out codecademy's 
\href{http://www.codecademy.com/en/tracks/python}{Python tutorial}.

\section{Environment}

It is necessary you have a proper development environment set up before
beginning this assignment. You will need Python 3, Node.js and MongoDB. Your npm
version should be up to date (Node.js comes with npm 2.7.4 but the latest is
2.9.something). Upgrading npm can be achieved with \\

\noindent \texttt{\$ npm install -g npm} \\

If the previous does not upgrade your npm, it is most likely that you are on
Windows and the path to Node.js occurs before the path to node\_modules between
your PATH and path environment variables (PATH loaded first).  Once npm is up to
date, you can install a few other global modules which you will be using:

\begin{verbatim}
$ npm install -g yo bower grunt-cli gulp
\end{verbatim}

If you have issues starting a local MongoDB service, again, you are probably on
Windows and what you need to do is 

\begin{verbatim}
$ mkdir C:\data\db    
\end{verbatim}

and run (while you start \texttt{mongo} in another terminal) 

\begin{verbatim}
$ mongodb -dbpath C:\data\db  
\end{verbatim}
If you have any issues getting set up, Google is your friend. 

\section{Requirements}

Your app must:
\begin{itemize}
    \item use a Python script to convert DNA $\rightarrow$ protein
    \item use JavaScript to convert protein $\rightarrow$ DNA
    \item accept data at endpoints of your choice
    \item provide endpoints to retreive data from database
    \item use unit testing with Mocha
    \item store converted data in the database 
\end{itemize}

\section{Boilerplates}

You may find the following boilerplate generators useful.

\noindent The generator made by the Express team: \\
\texttt{\$ npm install -g express-generator} \\ 
A port of the above to yeoman: \\
\texttt{\$ npm insall -g generator-express} \\
A popular MEAN boilerplate: \\
\texttt{\$ npm install -g generator-angular-fullstack} \\
mean.io's generator \\
\texttt{\$ npm install -g mean-cli} \\
A front-end only AngularJS generator \\
\texttt{\$ npm install -g generator-gulp-angular}

\section{The MEAN Stack}

The MEAN stack is the acronym given to a full-stack web server using Node.js as
it's serverside engine, with Express as a framework for Node, using MongoDB for
a database, and using the frontend framework AngularJS which eases the 
difficulty of making single page applications. We will go through a quick intro
to each of these members, but first an understanding of how a web application
works must be acquired. 

A simple, bare bones website is just an \texttt{html} file, or series of
\texttt{html} files. HTML stands for \textit{hypertext markup language} and is 
very simple to understand. Here is a basic \texttt{helloworld.html} webpage:

%\lstlisting{files/html/helloworld.html}

\inputminted{html}{files/html/helloworld.html}







\end{document}
